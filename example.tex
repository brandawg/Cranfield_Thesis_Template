\documentclass[12pt]{book}

% \usepackage[backend=bibtex, sorting=none, style=authoryear-ibid, citestyle=authoryear]{biblatex}
% \addbibresource{BibTexCollection.bib}

\usepackage{blindtext}
\usepackage{amsmath}
\usepackage{amssymb}
\usepackage{graphicx}
\usepackage{xfrac}

% Place style file after other packages.
\usepackage{cranfieldthesis2}

% Additional set up (not necessary)
\itemsep0em 
\usepackage [autostyle, english = american]{csquotes}
\MakeOuterQuote{"}

% Title Page Set Up
\title{Thesis Title}
\author{Christian Name \ Surname}
\date{April 2019}
\school{Defence and Security}
\centre{Centre for Electronic Warfare, Information and Cyber}
\degree{Ph.D}
\academicyear{2019}
\supervisor{Supervisor}
\copyrightyear{2019}
\fulfilment{}

\arialFont

\begin{document}

% Standard-Form Title Pages
\maketitlepages

% Abstract and Keywords
\begin{abstract}
    \blindtext
    \section*{Keywords}
    Keyword 1; keyword 2; keyword 3.
\end{abstract}

% Table of Contents
\sstableofcontents

% List of Figures
\sslistoffigures

% List of Tables
\sslistoftables

% List of Code
\sslistofcode

% The list of abbreviations can't be automatically generated so you need to populate it yourself
\begin{listofabbreviations}
    \abbrev{EWIC}{Centre for Electronic Warfare, Information and Cyber}
\end{listofabbreviations}

\begin{listofsymbols}
	\symb{$A$}{$m$}{Antenna, cartesian coordinate (transceiver).}
\end{listofsymbols}

\chapter{First Chapter}\label{chap:one}

Use the cleveref package for referencing; e.g. reference chapter: \Cref{chap:one}

\begin{figure}[h!]
  \centering
  \rule{20pt}{20pt}
  \caption{My figure. Left algined text is set in the style file, so you can write lots of text and still have the figure caption look professional.}
  \label{fig:myfig1}
\end{figure}

\noindent Reference lowercase figure: \cref{fig:myfig1}\\
\noindent Reference uppercase figure: \Cref{fig:myfig1}

\begin{equation} 
  x = \sqrt{\frac{1}{M N}\sum_{i=0}^{N-1}\sum_{j=0}^{M - 1}(I_{ij} - \bar{I})^2} \label{eq:myEq}
\end{equation}

\noindent Reference lowercase equation: \cref{eq:myEq}
\noindent Reference uppercase equation: \Cref{eq:myEq}

\section{New Section}

\begin{equation}
x = 2
\end{equation}

\blindtext
\blindtext
\begin{align}
  y&=a_1x+b_1\label{eq:1}\\
  y&=a_2x+b_2\label{eq:2}\\
  y&=a_3x+b_3\label{eq:3}\\
  y&=a_4x+b_4\label{eq:4}
\end{align}

\noindent
Range example: \crefrange{eq:1}{eq:4}
 
\begin{figure}[ht]
  \centering
  \rule{0.5\linewidth}{0.25\linewidth}
  \caption{Second figure}
  \label{fig:myfig2}
\end{figure}
 
\noindent
Mixed references example: \cref{eq:1,eq:3,eq:4,fig:myfig2}

\subsection{Sub-Section}
\blindtext 

Table Example - \cref{tab:mytable}.

\begin{table}[h!]
  \centering
  \begin{tabular}{ccc}
  \hline
  \textbf{Col 1} & \textbf{Col 2} & \textbf{Col 3} \\ \hline
  1              & 4              & 7              \\ \hline
  2              & 5              & 8              \\ \hline
  3              & 6              & 9              \\ \hline
  \end{tabular}
  \caption{Here is my first table.}
  \label{tab:mytable}
\end{table}


\chapter{New Chapter}

\section{Section}

% \printbibliography[title=\uppercase{references}]

\appendix

\chapter{Appendix Chapter}

\section{Section}

\end{document}